% !TEX options=--shell-escape
\title{Summary - QRM}
\author{
        Pascal Müller [pamuelle@student.ethz.ch]
}
\date{\today}

\documentclass[12pt]{article}
\usepackage{etoolbox}
\usepackage[utf8]{inputenc}
\usepackage[parfill]{parskip}
\usepackage{amsmath}
\usepackage{xcolor}
\usepackage{physics}
\usepackage[margin=0.5in]{geometry}
\usepackage{multicol}
\usepackage{fourier}
\usepackage{graphicx}
\graphicspath{ {./img/} }
% Make all the math blue - uses eetoolbox package
\everymath{\color{blue!70}}
\AtBeginEnvironment{equation*}{\color{blue!70}}
\AtBeginEnvironment{equation}{\color{blue!70}}

\definecolor{bkg_light}{HTML}{d0ddf2}
\definecolor{bkg_dark}{HTML}{717ebf}

\newcommand{\mysection}[1]{
    \setlength\fboxsep{4pt} %% spacing around box contents
    \section*{\colorbox{bkg_dark}{\makebox[0.47\textwidth]{\color{white}#1}}}
}

\newcommand{\mysubsection}[1]{
    \setlength\fboxsep{4pt} %% spacing around box contents
    \subsection*{\colorbox{bkg_light}{\makebox[0.47\textwidth]{\color{black}#1}}}
}

%\usepackage[outputdir=/home/pascal/.config/sublime-text-3/Cache]{minted}

\begin{document}
\maketitle

Serien zusammengefasst: 1 \\
Slides zusammengefasst: 1 

\textbf{Notes:} \\
Slide 1: Vieles ausgelassen, falls Platz, gehe nochmal über
Slide 1. \\
Slide 2: Lots of random math I didnt include yet. Waiting on the rest of the
course.


TODO:
\begin{itemize}
  \item Add different types of risk (Slide 1)
  \item Add Basel accords?
  \item Add Solvency I and II?
  \item Add Standard formula and internal model approach?
  \item ADD series 2 aufgabe 1.a auf zsmfg resp. gauss integral - siehe
        stackoverflow and integral solver.
  \item !Lesbesque integral trick thingie. See slide 2, S2A2f
  \item Check S3 notes on what to add to the summary.
  \item Student-t See S3A4
  \item Add normal dist integral (substitution)
  \item VaR and ES von allen distributions.
\end{itemize}

\begin{multicols*}{2}

\mysection{Definitions \& Concepts}
\mysubsection{Main Goal of Regulation}
Ensure that financial institutions have enough capital to remain solvent.

\mysubsection{Three Pillars of Financial Regulation}
The three pillar concept is at the basis of the Basel II and Solvency II
regulatory frameworks.

\paragraph{Pillar 1: Minimal Capital Charge} Requirement for the calculation of
the regulatory capital to ensure that a bank holds sufficient capital for its
\textcolor{green}{credit risk} in the \textit{banking book},
\textcolor{green}{market risk} in the \textit{trading book} and
\textcolor{green}{operational risk}.

\paragraph{Pillar 2: Supervisory Review Process} Local regulators review the
checks and balances put in place for capital adequacy assessments, ensure that

\mysubsection{Model Uncertainity}
Model uncertainty refers to the uncertainty about the accuracy of a model. It
results from imprecise and idealized assumptions, which, to some degree, have
to be made in every modeling framework.

\mysubsection{Knightian Uncertainity}
In economics, Knightian uncertainty is a lack of any quantifiable knowledge 
about some possible occurrence, as opposed to the presence of quantifiable
risk. The concept acknowledges some fundamental degree of ignorance, a limit 
to knowledge, and an essential unpredictability of future events.

\mysubsection{Ambiguity Aversion}
In decision theory and economics, ambiguity aversion is a preference for known 
risks over unknown risks. An ambiguity-averse individual would rather choose 
an alternative where the probability distribution of the outcomes is known 
over one where the probabilities are unknown.

TODO: Pic

\mysubsection{Empirical Distribution Function}
Let $x_1, \dots, x_n$ be independent realizations of a random variable $X$. The
corresponding \textit{empirical distribution function} 
$\hat{F}_X : \mathbb{R} \to [0, 1]$ is given by the step function
\[
  \hat{F}_X(x) = \frac{1}{n} \sum_{i=1}^n 1_{\{x_i \leq x\}} \qquad x\in
  \mathbb{R}
\]
whereas $1_{\{\cdot\}}$ is the indicator function.

\paragraph{An application: Generating Loss Distr.}
We can approximate the dist. of $L_{t+1}$ by the edf above:
\[
  \hat{F}_X(x) = \frac{1}{n} \sum_{i=1}^n 1_{\{\textcolor{red}{l_{l-i+1}}
  \leq x\}}
\]
where $l_{t-n+1}, \dots, l_t$ are the last $n$ realized losses. Each of those
losses has equal probability (hence the $1/n$ and the indicator function).

\paragraph{Advantage} Easy to implement, no modeling assumptions, no eastimation
required (you just look at past losses).

\paragraph{Drawbacks} Sufficient data for all risk-factors required, makes
predictions based on past data.

% Cummulative Distribution Function
\mysubsection{Cummulative Distribution Function}
Let $X$ be a random variable. The CDF $F_X : \mathbb{R} \to [0, 1]$ given by
$\mathbf{F_X(x) = \mathbb{P}[X \leq x]}$ satisfies
\begin{alignat*}{2}
  &\text{Normalization} &&\lim_{x \to -\infty} F_X(x) = 0 \text{ and }
    \lim_{x \to \infty}F_X(x) = 1 \\
  &\text{Right-Continuity} \quad &&F_X(x_n) \downarrow F_X(x) \text{ for } x_n
  \downarrow x \in \mathbb{R} \\
  &\text{Monotonicity} &&F_X(a) \leq F_L(b) \text{ for } a \leq b
\end{alignat*}

\mysubsection{Expectation Value}
$\mathbb{E}[X] = \sum_{i=1}^\infty x_i p_i$

$\mathbb{E}[X] = \int_{-\infty}^\infty x f(x) dx$

\warning $\mathbb{E}$ is linear: $\mathbb{E}[\sum_{i=1}^N a_i X_i]
= \sum_{i=1}^N a_i \mathbb{E}[X_i]$

\mysubsection{Variance}
$Var(X) = \mathbb{E}[(X - \mathbb{E}[X])^2]
        = \mathbb{E}[X^2] - \mathbb{E}[X]^2$

\mysubsection{Value-at-Risk $\text{VaR}_\alpha(X)$}
\warning Basically a quantile.
\[
  \text{VaR}_\alpha(X) = \min\{m\in\mathbb{R} : \mathbb{P}[X - m \leq 0]
  \geq \alpha\}
\]
\paragraph{Korollar:} Let $L$ be rnd. var taking values in interval $I \subset
\mathbb{R}$. Then
\[
  \text{VaR}_\alpha(f(L)) = f(\text{VaR}_\alpha(L))
\]
for each $\alpha \in (0, 1)$ and every strictly increasing continous function
$f: I \to \mathbb{R}$.

TODO: Intuition, Pic


\mysubsection{Conditional Expectation}
\[
  \mathbb{E}[X|X \textcolor{red}{\geq} \mu] = \frac{1}{\textcolor{red}
  {\mathbb{P}}[X \geq \mu]} \int_\mu^\infty xf(x)dx
\]
with $\mathbb{P}[X\geq\mu] = \int_\mu^\infty f(x)dx$
\warning See Korollar below.
\mysubsection{Expected Shortfall $\text{ES}_\alpha(X)$}
\[
  \text{ES}_\alpha(X) = \mathbb{E}[X | X \geq \text{VaR}_\alpha(X)]
\]
\paragraph{Korollar:} Let $L$ be a rnd var. and $a\in \mathbb{R}$ s.t.
$\mathbb{P}[L \geq 0]$. Then:
$\mathbb{E}[L | L \geq a] = a + \frac{1}{\mathbb{P}[L \geq a]}
\int_a^\infty \mathbb{P}[L \geq x] dx$

\warning $\mathbb{P}[L \geq a]$ is the tail, probably equal $1-\alpha$.
TODO: Is the warning always true? Always $1-\alpha$?

TODO: Intuition, Pic





\newpage
%
% PROBABILITY
%
\mysection{Probability}
\mysubsection{Basics}
$F_X(x) = \mathbb{P}[X \leq x] = \int_{-\infty}^x f(x') dx'$

\mysubsection{Bayes Theorem}
$P(A | B) = \frac{P(B | A)P(A)}{P(B)}$
\mysubsection{Law of Total Probability}
$P(A) = \sum_n P(A | B_n) P(B_n)$ where $B_n$ discrete and finite.
\paragraph{Example:} $Y \sim Bernoulli$ and $Z \sim Exp(\lambda)$. \\
$\mathbb{P}[YZ \leq x] = \mathbb{P}[YZ \leq x | Y = 1]\mathbb{P}[Y=1]
+ \mathbb{P}[YZ \leq x | Y = 0]\mathbb{P}[Y = 0]$

%
% DISTRIBUTIONS
%
\mysection{Distributions}
% Normal distribution
\mysubsection{Normal}
\warning Can NOT use Gaussian Integral for Gaussian Functions!
\warning Use a sub for computations!

TODO: See S2A1. Note: See the question I asked in the mathematics Discord.

\mysubsection{Bernoulli}
\warning Takes value $1$ with prob. $p$ and $0$ with prob. $q = 1 - p$.
\warning Binomial Dist. with $n=1$.

\[
  f(k;p) = \begin{cases}
    p & k = 1 \\
    q = 1-p & k = 0
  \end{cases}
\]
$f(k;p) = p^k(1-p)^{1-k} = pk + (1-p)(1-k)$ for $k \in (0, 1)$ 

$\mathbb{E}[X] = Pr(X = 1) \cdot 1 + Pr(X = 0)\cdot 0 = p$ \\
$\mathbb{E}[X^2] = p = \mathbb{E}[X]$ \\
$Var[X] = pq = p(1-p)$


% Poisson
\mysubsection{Poisson: Pois($\lambda$)}
\warning When computing $\mathbb{E}[N^2]$ use factorial trick.

\[
  \mathbb{P}[N = n] = e^{-\lambda}\frac{\lambda^n}{n!}
  \quad n = \textcolor{red}{0}, 1, 2, \dots
\]

$\mathbb{E}[N] = \lambda$ (Use Series for exp.)

$\mathbb{E}[N^{\textcolor{red}{2}}] = \lambda^2 + \mathbb{E}[N] = \lambda^2 +
  \lambda$

$Var[N] = \lambda = \mathbb{E}[N]$

% Exponential
\mysubsection{Exponential: $\text{Exp}(\lambda)$}

PDF: $f(x) = \begin{cases}
                \lambda e^{-\lambda x} & x \geq 0 \\ 0 & x < 0 
             \end{cases}
             \qquad \lambda > 0
             $

$\mathbb{E}[X] = \frac{1}{\lambda}$

$\mathbb{E}[X^2] = \frac{2}{\lambda^2}$

$Var[X] = \frac{1}{\lambda^2}$

\paragraph{Example: Compute dist. of $exp(X)$} Denote $Y = \exp(X)$ and $F_Y(y) =
\mathbb{P}[Y \leq y]$. Since $x \geq 0 \ \Leftrightarrow \ y = \exp(x) \geq 1$,
one has $F_Y(y) = 0$ for $y < 1$, whereas
\begin{align*}
  F_Y(y) &= \mathbb{P}[\exp(X) \leq y] = \mathbb{P}[X \leq \log{y}] \\
  &= \int_0^{\log{y}} \lambda e^{-\lambda x}dx = 1 - y^{-\lambda}
\end{align*}
for $y \geq 1$. Computing $f(x) = \frac{d}{dy} F_Y(y)$ we get the density
\[
  f(y) = \begin{cases}
           \lambda y^{-y - 1} = \frac{\lambda}{y^{y+1}} & y \geq 1 \\
           0 & y < 1
         \end{cases}
\]
Hence, $Y$ has a Pareto distribution with scale parameter 1 and shape parameter
$\lambda$.

\mysubsection{Pareto}
TODO: See S1A4.c

\newpage

\mysection{Mathematical Tools}
\mysubsection{Partial Integration}
\[
  \int_a^b u v' dx = [u v]^b_a - \int_a^b u' v dx
\]

\mysubsection{Tricks}
\paragraph{Factorial Trick}
\[
  \frac{n^2}{n!} = \frac{n(n-1) + n}{n!} = \frac{1}{(n-2)!} + \frac{1}{(n-1)!}
\]

\mysubsection{Series}

\[
  e^\lambda = \sum_{n = \textcolor{red}{0}}^\infty \frac{\lambda^n}{n!}
\]

\end{multicols*}

\end{document}
